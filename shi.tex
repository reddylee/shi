% Options for packages loaded elsewhere
\PassOptionsToPackage{unicode}{hyperref}
\PassOptionsToPackage{hyphens}{url}
%
\documentclass[
]{book}
\usepackage{amsmath,amssymb}
\usepackage{lmodern}
\usepackage{ifxetex,ifluatex}
\ifnum 0\ifxetex 1\fi\ifluatex 1\fi=0 % if pdftex
  \usepackage[T1]{fontenc}
  \usepackage[utf8]{inputenc}
  \usepackage{textcomp} % provide euro and other symbols
\else % if luatex or xetex
  \usepackage{unicode-math}
  \defaultfontfeatures{Scale=MatchLowercase}
  \defaultfontfeatures[\rmfamily]{Ligatures=TeX,Scale=1}
\fi
% Use upquote if available, for straight quotes in verbatim environments
\IfFileExists{upquote.sty}{\usepackage{upquote}}{}
\IfFileExists{microtype.sty}{% use microtype if available
  \usepackage[]{microtype}
  \UseMicrotypeSet[protrusion]{basicmath} % disable protrusion for tt fonts
}{}
\makeatletter
\@ifundefined{KOMAClassName}{% if non-KOMA class
  \IfFileExists{parskip.sty}{%
    \usepackage{parskip}
  }{% else
    \setlength{\parindent}{0pt}
    \setlength{\parskip}{6pt plus 2pt minus 1pt}}
}{% if KOMA class
  \KOMAoptions{parskip=half}}
\makeatother
\usepackage{xcolor}
\IfFileExists{xurl.sty}{\usepackage{xurl}}{} % add URL line breaks if available
\IfFileExists{bookmark.sty}{\usepackage{bookmark}}{\usepackage{hyperref}}
\hypersetup{
  pdftitle={耕读诗经},
  pdfauthor={秦耕读},
  hidelinks,
  pdfcreator={LaTeX via pandoc}}
\urlstyle{same} % disable monospaced font for URLs
\usepackage{color}
\usepackage{fancyvrb}
\newcommand{\VerbBar}{|}
\newcommand{\VERB}{\Verb[commandchars=\\\{\}]}
\DefineVerbatimEnvironment{Highlighting}{Verbatim}{commandchars=\\\{\}}
% Add ',fontsize=\small' for more characters per line
\usepackage{framed}
\definecolor{shadecolor}{RGB}{248,248,248}
\newenvironment{Shaded}{\begin{snugshade}}{\end{snugshade}}
\newcommand{\AlertTok}[1]{\textcolor[rgb]{0.94,0.16,0.16}{#1}}
\newcommand{\AnnotationTok}[1]{\textcolor[rgb]{0.56,0.35,0.01}{\textbf{\textit{#1}}}}
\newcommand{\AttributeTok}[1]{\textcolor[rgb]{0.77,0.63,0.00}{#1}}
\newcommand{\BaseNTok}[1]{\textcolor[rgb]{0.00,0.00,0.81}{#1}}
\newcommand{\BuiltInTok}[1]{#1}
\newcommand{\CharTok}[1]{\textcolor[rgb]{0.31,0.60,0.02}{#1}}
\newcommand{\CommentTok}[1]{\textcolor[rgb]{0.56,0.35,0.01}{\textit{#1}}}
\newcommand{\CommentVarTok}[1]{\textcolor[rgb]{0.56,0.35,0.01}{\textbf{\textit{#1}}}}
\newcommand{\ConstantTok}[1]{\textcolor[rgb]{0.00,0.00,0.00}{#1}}
\newcommand{\ControlFlowTok}[1]{\textcolor[rgb]{0.13,0.29,0.53}{\textbf{#1}}}
\newcommand{\DataTypeTok}[1]{\textcolor[rgb]{0.13,0.29,0.53}{#1}}
\newcommand{\DecValTok}[1]{\textcolor[rgb]{0.00,0.00,0.81}{#1}}
\newcommand{\DocumentationTok}[1]{\textcolor[rgb]{0.56,0.35,0.01}{\textbf{\textit{#1}}}}
\newcommand{\ErrorTok}[1]{\textcolor[rgb]{0.64,0.00,0.00}{\textbf{#1}}}
\newcommand{\ExtensionTok}[1]{#1}
\newcommand{\FloatTok}[1]{\textcolor[rgb]{0.00,0.00,0.81}{#1}}
\newcommand{\FunctionTok}[1]{\textcolor[rgb]{0.00,0.00,0.00}{#1}}
\newcommand{\ImportTok}[1]{#1}
\newcommand{\InformationTok}[1]{\textcolor[rgb]{0.56,0.35,0.01}{\textbf{\textit{#1}}}}
\newcommand{\KeywordTok}[1]{\textcolor[rgb]{0.13,0.29,0.53}{\textbf{#1}}}
\newcommand{\NormalTok}[1]{#1}
\newcommand{\OperatorTok}[1]{\textcolor[rgb]{0.81,0.36,0.00}{\textbf{#1}}}
\newcommand{\OtherTok}[1]{\textcolor[rgb]{0.56,0.35,0.01}{#1}}
\newcommand{\PreprocessorTok}[1]{\textcolor[rgb]{0.56,0.35,0.01}{\textit{#1}}}
\newcommand{\RegionMarkerTok}[1]{#1}
\newcommand{\SpecialCharTok}[1]{\textcolor[rgb]{0.00,0.00,0.00}{#1}}
\newcommand{\SpecialStringTok}[1]{\textcolor[rgb]{0.31,0.60,0.02}{#1}}
\newcommand{\StringTok}[1]{\textcolor[rgb]{0.31,0.60,0.02}{#1}}
\newcommand{\VariableTok}[1]{\textcolor[rgb]{0.00,0.00,0.00}{#1}}
\newcommand{\VerbatimStringTok}[1]{\textcolor[rgb]{0.31,0.60,0.02}{#1}}
\newcommand{\WarningTok}[1]{\textcolor[rgb]{0.56,0.35,0.01}{\textbf{\textit{#1}}}}
\usepackage{longtable,booktabs,array}
\usepackage{calc} % for calculating minipage widths
% Correct order of tables after \paragraph or \subparagraph
\usepackage{etoolbox}
\makeatletter
\patchcmd\longtable{\par}{\if@noskipsec\mbox{}\fi\par}{}{}
\makeatother
% Allow footnotes in longtable head/foot
\IfFileExists{footnotehyper.sty}{\usepackage{footnotehyper}}{\usepackage{footnote}}
\makesavenoteenv{longtable}
\usepackage{graphicx}
\makeatletter
\def\maxwidth{\ifdim\Gin@nat@width>\linewidth\linewidth\else\Gin@nat@width\fi}
\def\maxheight{\ifdim\Gin@nat@height>\textheight\textheight\else\Gin@nat@height\fi}
\makeatother
% Scale images if necessary, so that they will not overflow the page
% margins by default, and it is still possible to overwrite the defaults
% using explicit options in \includegraphics[width, height, ...]{}
\setkeys{Gin}{width=\maxwidth,height=\maxheight,keepaspectratio}
% Set default figure placement to htbp
\makeatletter
\def\fps@figure{htbp}
\makeatother
\setlength{\emergencystretch}{3em} % prevent overfull lines
\providecommand{\tightlist}{%
  \setlength{\itemsep}{0pt}\setlength{\parskip}{0pt}}
\setcounter{secnumdepth}{5}
\usepackage{booktabs}
\ifluatex
  \usepackage{selnolig}  % disable illegal ligatures
\fi
\usepackage[]{natbib}
\bibliographystyle{apalike}

\title{耕读诗经}
\author{秦耕读}
\date{2022-01-17}

\usepackage{amsthm}
\newtheorem{theorem}{Theorem}[chapter]
\newtheorem{lemma}{Lemma}[chapter]
\newtheorem{corollary}{Corollary}[chapter]
\newtheorem{proposition}{Proposition}[chapter]
\newtheorem{conjecture}{Conjecture}[chapter]
\theoremstyle{definition}
\newtheorem{definition}{Definition}[chapter]
\theoremstyle{definition}
\newtheorem{example}{Example}[chapter]
\theoremstyle{definition}
\newtheorem{exercise}{Exercise}[chapter]
\theoremstyle{definition}
\newtheorem{hypothesis}{Hypothesis}[chapter]
\theoremstyle{remark}
\newtheorem*{remark}{Remark}
\newtheorem*{solution}{Solution}
\begin{document}
\maketitle

{
\setcounter{tocdepth}{1}
\tableofcontents
}
\hypertarget{ux8bf4ux660e}{%
\chapter{说明}\label{ux8bf4ux660e}}

\hypertarget{ux6bdbux8bd7ux5927ux5e8f}{%
\section{毛诗大序}\label{ux6bdbux8bd7ux5927ux5e8f}}

《毛诗序》,古代中国诗歌理论著作。一说为孔丘弟子子夏作,一说为汉人卫宏为《诗经》所作的序,分为大序和小序。大序为《关雎》题解之后作者所作的全部《诗经》的总的序言,小序是诗经三百零五篇中每一篇的序言。一般而言《毛诗序》是指大序。
\textgreater{}``诗者,志之所之也。在心为志,发言为诗。情动于中,而行于言。言之不足,故嗟叹之。嗟叹之不足,故永歌之。永歌之不足,不知手之舞之,足之蹈之也。情发于声,声成文谓之音。治世之音安以乐,其政和;乱世之音怨以怒,其政乖;亡国之音哀以思,其民困。故正得失,动天地,感鬼神,莫近于诗。先王以是经夫妇,成孝敬,厚人伦,美教化,移风俗。故《诗》有六义焉:一曰风,二曰赋,三曰比,四曰兴,五曰雅,六曰颂。上以风化下,下以风刺上,主文而谲谏,言之者无罪,闻之者足以戒,故曰风。至于王道衰,礼义废,政教失,国异政,家殊俗,而变《风》变《雅》作矣。国史明乎得失之迹,伤人伦之废,哀刑政之苛,吟咏情性以风其上,达于事变而怀其旧焉。《颂》者,美盛德之形容,以其成功告于神明者也。是谓``四始'',诗之至也。"

作者不详,约成书于西汉,很可能经过东汉经学家卫宏修改。关于《毛诗序》的作者,历来众说纷纭,尚无定论。

\hypertarget{ux6bdbux8bd7ux6b63ux4e49ux5e8f--ux5510ux5b54ux9896ux8fbe574648}{%
\section{《毛诗正义》序- 唐·孔颖达(574---648)}\label{ux6bdbux8bd7ux6b63ux4e49ux5e8f--ux5510ux5b54ux9896ux8fbe574648}}

\begin{quote}
``夫诗者,论功颂德之歌,止僻防邪之训。虽无为而自发,乃有益于生灵。六情静于中,百物荡于外。情缘物动,物感情迁。若政遇醇和,则欢娱被于朝野;时当墋黩,亦怨刺形于咏歌。作之者所以畅怀舒愤,闻之者足以塞违从正。发诸情性,谐于律吕。故曰感天地,动鬼神,莫近于诗。此乃诗之为用,其利大矣。若夫哀乐之起,冥于自然;喜怒之端,非由人事。故燕雀表啁噍之感,鸾凤有歌舞之容。然则诗理之先,同夫开辟,诗迹所用,随运而移。上皇道质,故讽谕之情寡;中古政繁,亦讴歌之理切。唐虞乃见其初,牺轩莫测其始。于后时经五代,篇有三千,成康没而颂声寝,陈灵兴而变风息。先君宣父,釐正遗文,缉其精华,褫其烦重,上从周始,下暨鲁僖,四百年间,六诗备矣。卜商阐其业,雅颂与金石同和;秦正燎其书,简牍与烟尘共尽。汉氏之初,诗分为四。申公腾芳于鄢郢,毛诗光价于河间,贯长卿传之于前,郑康成笺之于后。晋宋二萧之世,其道大行;齐魏两河之间,兹风不坠。其近代为义疏者,有全缓何允舒援刘轨思刘鬼刘绰刘炫等,然绰炫并聪颖特达,文而又儒,擢秀干于一时,骋绝辔于千里,固诸儒之所揖让,日下之无双,于其所作疏内,特为殊绝。今奉敕删定,故据以为本。然绰炫等负恃才气,轻鄙先达,同其所异,异其所同,或应略而反详,或宜详而更略。准其绳墨,差忒未免,勘其会同,时有颠踬。今则削其所烦,增其所简,唯意存于曲直,非有心于爱憎,谨与朝散大夫行太学博士臣王德韶、徵仕郎守四门博士臣齐威等对共讨论,辨详得失。至十六年,又奉敕与前修疏人及给事郎守太学助教云骑尉臣赵乾叶、登仕郎守四门助教云骑尉臣贾普耀等对使赵宏智覆更详正,凡为四十卷。庶以对扬圣范,垂训幼蒙。故序其所见,载之于卷首云尔。''
\end{quote}

\hypertarget{ux56fdux98ce}{%
\chapter*{国风}\label{ux56fdux98ce}}
\addcontentsline{toc}{chapter}{国风}

\hypertarget{ux56fdux98ceux91ca}{%
\section{国风释}\label{ux56fdux98ceux91ca}}

朱熹曰:
国者,诸侯所封之域,而风者,民俗歌谣之诗也。谓之风者,以其被上之化以有言,而其言又足以感人,如物因风之动以有声,而其声又足以动物也。是以诸侯采之以贡于天子,天子受之而列于乐官,于以考其俗尚之美恶,而知其政治之得失焉。旧说《二南》为正风,所以用之闺门、乡党、邦国,而化天下也。十三国为变风,则亦领在乐官,以时存肄,备观省而垂监戒耳。合之凡十五国云。

\hypertarget{ux5468ux5357-ux5173ux96ce}{%
\section{周南-关雎}\label{ux5468ux5357-ux5173ux96ce}}

\hypertarget{ux5468ux5357ux89e3}{%
\subsection{周南解}\label{ux5468ux5357ux89e3}}

周,国名。南,南方诸侯之国也。周国本在《禹贡》雍州境内岐山之阳。后稷十三世孙古公亶父始居其地,传子王季历,至孙文王昌,辟国浸广。于是徙都于丰,而分岐周故地以为周公旦、召公奭之采邑,且使周公为政于国中,而召公宣布于诸侯。于是德化大成于内,而南方诸侯之国,江、沱、汝、汉之间,莫不从化,盖三分天下而有其二焉。至子武王发,又迁于镐,遂克商而有天下。武王崩,子成王诵立。周公相之,制作礼乐,乃采文王之世风化所及民俗之诗,被之管弦,以为房中之乐,而又推之以及于乡党邦国。所以著明先王风俗之盛,而使天下后世之修身、齐家、治国、平天下者,皆得以取法焉。盖其得之国中者,杂以南国之诗,而谓之《周南》。言自天子之国而被于诸侯,不但国中而已也。其得之南国者,则直谓之《召南》。言自方伯之国被于南方,而不敢以系于天子也。岐周,在今凤翔府岐山县。丰在今京兆府鄠县终南山北。南方之国,即今兴元府、京西、湖北等路诸州。镐在丰东二十五里。小序曰:``《关雎》、《麟趾》之化,王者之风,故系之周公。南,言化自北而南也。《鹊巢》、《驺虞》之德,诸侯之风也,先王之所以教,故系之召公。''斯言得之矣。

Excerpt From
诗经
[宋]朱熹集传
This material may be protected by copyright.

\hypertarget{ux9898ux89e3}{%
\subsection{题解}\label{ux9898ux89e3}}

写``君子''思慕``淑女''的心情,并想象得到她以后的快乐。

\hypertarget{ux8bd7ux6587}{%
\subsection{诗文}\label{ux8bd7ux6587}}

关关雎鸠 在河之洲
窈窕淑女 君子好逑
参差荇菜 左右流之
窈窕淑女 寤寐求之
求之不得 寤寐思服
悠哉悠哉 辗转反侧
参差荇菜 左右采之
窈窕淑女 琴瑟友之
参差荇菜 左右芼之
窈窕淑女 钟鼓乐之

\hypertarget{ux6ce8ux91ca}{%
\subsection{注释}\label{ux6ce8ux91ca}}

关关:形容水鸟的和鸣声。 雎鸠:一种水鸟。 洲:河中沙洲。
窈窕:娴静美好的样子。 淑:善也。 好逑:佳偶。
参差:长短不齐。 荇菜:一种水草,可食。
流:通``捞''。
寤寐求之:形容日思夜想。寤,醒来。寐,睡着。
思服:思念。
悠哉:思念之深长。 辗转反侧:形容睡不着。
琴瑟:乐器,动词用,或喻和谐。 友:亲密、亲近。
芼:通``摸'',水下摸索荇菜之意。
乐:娱悦,或通``撩'',以音乐追求女子的方式。

\hypertarget{ux8015ux8bfbux8bf4}{%
\subsection{耕读说}\label{ux8015ux8bfbux8bf4}}

\hypertarget{ux845bux8983}{%
\section{葛覃}\label{ux845bux8983}}

\hypertarget{ux9898ux89e3-1}{%
\subsection{题解}\label{ux9898ux89e3-1}}

《诗序》:
《葛覃》,后妃之本也。后妃在父母家,则志在于女功之事,躬俭节用,服浣濯之衣,尊敬师傅,则可以归安父母,化天下以妇道也。

\hypertarget{ux8bd7ux6587-1}{%
\subsection{诗文}\label{ux8bd7ux6587-1}}

葛之覃兮,施于中谷;
维叶萋萋。黄鸟于飞,
集于灌木,其鸣喈喈。

葛之覃兮,施于中谷,
维叶莫莫。是刈是濩,
为絺为绤,服之无斁。

言告师氏,言告言归。
薄污我私,薄浣我衣。
害浣害否,归宁父母。

\hypertarget{ux6ce8ux91ca-1}{%
\subsection{注释}\label{ux6ce8ux91ca-1}}

{[}1{]}赋也。葛,草名,蔓生,可为絺绤者。覃,延。施,移也。中谷,谷中也。萋萋,盛貌。黄鸟,鹂也。灌木,丛木也。喈喈,和声之远闻也。〇赋者,敷陈其事而直言之者也。盖后妃既成絺绤,而赋其事,追叙初夏之时,葛叶方盛,而有黄鸟鸣于其上也。后凡言赋者放此。

[2]赋也。莫莫,茂密貌。刈,斩。濩,煮也。精曰絺,粗曰绤。斁,厌也。〇此言盛夏之时,葛既成矣,于是治以为布,而服之无厌。盖亲执其劳,而知其成之不易,所以心诚爱之,虽极垢弊,而不忍厌弃也。

[3]赋也。言,辞也。师,女师也。薄,犹少也。污,烦撋之以去其污,犹治乱而曰乱也。浣则濯之而已。私,燕服也。衣,礼服也。害,何也。宁,安也,谓问安也。
上章既成絺绤之服矣,此章遂告其师氏,使告于君子以将归宁之意。且曰:盍治其私服之污,而浣其礼服之衣乎?何者当浣,而何者可以未浣乎?我将服之以归宁于父母矣。

\hypertarget{ux8015ux8bfbux8bf4-1}{%
\subsection{耕读说}\label{ux8015ux8bfbux8bf4-1}}

\hypertarget{hello-bookdown}{%
\chapter{Hello bookdown}\label{hello-bookdown}}

All chapters start with a first-level heading followed by your chapter title, like the line above. There should be only one first-level heading (\texttt{\#}) per .Rmd file.

\hypertarget{a-section}{%
\section{A section}\label{a-section}}

All chapter sections start with a second-level (\texttt{\#\#}) or higher heading followed by your section title, like the sections above and below here. You can have as many as you want within a chapter.

\hypertarget{an-unnumbered-section}{%
\subsection*{An unnumbered section}\label{an-unnumbered-section}}
\addcontentsline{toc}{subsection}{An unnumbered section}

Chapters and sections are numbered by default. To un-number a heading, add a \texttt{\{.unnumbered\}} or the shorter \texttt{\{-\}} at the end of the heading, like in this section.

\hypertarget{ux5c0fux96c5}{%
\chapter*{小雅}\label{ux5c0fux96c5}}
\addcontentsline{toc}{chapter}{小雅}

\hypertarget{ux9e7fux9e23}{%
\section*{鹿鸣}\label{ux9e7fux9e23}}
\addcontentsline{toc}{section}{鹿鸣}

\hypertarget{ux8bd7ux5e8f}{%
\subsection*{《诗序》}\label{ux8bd7ux5e8f}}
\addcontentsline{toc}{subsection}{《诗序》}

《鹿鸣》,燕群臣嘉宾也。既饮食之,又实币帛筐篚以将其厚意,然后忠臣嘉宾得尽其心矣。

\hypertarget{ux8bd7ux6587-2}{%
\subsection*{诗文}\label{ux8bd7ux6587-2}}
\addcontentsline{toc}{subsection}{诗文}

``呦呦鹿鸣,食野之苹。
我有嘉宾,鼓瑟吹笙。
吹笙鼓簧,承筐是将。
人之好我,示我周行。

呦呦鹿鸣,食野之蒿。
我有嘉宾,德音孔昭。
视民不恌,君子是则是效。
我有旨酒,嘉宾式燕以敖。

呦呦鹿鸣,食野之芩。
我有嘉宾,鼓瑟鼓琴。
鼓瑟鼓琴,和乐且湛。
我有旨酒,以燕乐嘉宾之心。

\hypertarget{ux6ce8ux91ca-2}{%
\subsection{注释}\label{ux6ce8ux91ca-2}}

\hypertarget{parts}{%
\chapter{Parts}\label{parts}}

You can add parts to organize one or more book chapters together. Parts can be inserted at the top of an .Rmd file, before the first-level chapter heading in that same file.

Add a numbered part: \texttt{\#\ (PART)\ Act\ one\ \{-\}} (followed by \texttt{\#\ A\ chapter})

Add an unnumbered part: \texttt{\#\ (PART\textbackslash{}*)\ Act\ one\ \{-\}} (followed by \texttt{\#\ A\ chapter})

Add an appendix as a special kind of un-numbered part: \texttt{\#\ (APPENDIX)\ Other\ stuff\ \{-\}} (followed by \texttt{\#\ A\ chapter}). Chapters in an appendix are prepended with letters instead of numbers.

\hypertarget{footnotes-and-citations}{%
\chapter{Footnotes and citations}\label{footnotes-and-citations}}

\hypertarget{footnotes}{%
\section{Footnotes}\label{footnotes}}

Footnotes are put inside the square brackets after a caret \texttt{\^{}{[}{]}}. Like this one \footnote{This is a footnote.}.

\hypertarget{citations}{%
\section{Citations}\label{citations}}

Reference items in your bibliography file(s) using \texttt{@key}.

For example, we are using the \textbf{bookdown} package \citep{R-bookdown} (check out the last code chunk in index.Rmd to see how this citation key was added) in this sample book, which was built on top of R Markdown and \textbf{knitr} \citep{xie2015} (this citation was added manually in an external file book.bib).
Note that the \texttt{.bib} files need to be listed in the index.Rmd with the YAML \texttt{bibliography} key.

The \texttt{bs4\_book} theme makes footnotes appear inline when you click on them. In this example book, we added \texttt{csl:\ chicago-fullnote-bibliography.csl} to the \texttt{index.Rmd} YAML, and include the \texttt{.csl} file. To download a new style, we recommend: \url{https://www.zotero.org/styles/}

The RStudio Visual Markdown Editor can also make it easier to insert citations: \url{https://rstudio.github.io/visual-markdown-editing/\#/citations}

\hypertarget{blocks}{%
\chapter{Blocks}\label{blocks}}

\hypertarget{equations}{%
\section{Equations}\label{equations}}

Here is an equation.

\begin{equation} 
  f\left(k\right) = \binom{n}{k} p^k\left(1-p\right)^{n-k}
  \label{eq:binom}
\end{equation}

You may refer to using \texttt{\textbackslash{}@ref(eq:binom)}, like see Equation \eqref{eq:binom}.

\hypertarget{theorems-and-proofs}{%
\section{Theorems and proofs}\label{theorems-and-proofs}}

Labeled theorems can be referenced in text using \texttt{\textbackslash{}@ref(thm:tri)}, for example, check out this smart theorem \ref{thm:tri}.

\begin{theorem}
\protect\hypertarget{thm:tri}{}\label{thm:tri}For a right triangle, if \(c\) denotes the \emph{length} of the hypotenuse
and \(a\) and \(b\) denote the lengths of the \textbf{other} two sides, we have
\[a^2 + b^2 = c^2\]
\end{theorem}

Read more here \url{https://bookdown.org/yihui/bookdown/markdown-extensions-by-bookdown.html}.

\hypertarget{callout-blocks}{%
\section{Callout blocks}\label{callout-blocks}}

The \texttt{bs4\_book} theme also includes special callout blocks, like this \texttt{.rmdnote}.

You can use \textbf{markdown} inside a block.

\begin{Shaded}
\begin{Highlighting}[]
\FunctionTok{head}\NormalTok{(beaver1, }\AttributeTok{n =} \DecValTok{5}\NormalTok{)}
\CommentTok{\#\textgreater{}   day time  temp activ}
\CommentTok{\#\textgreater{} 1 346  840 36.33     0}
\CommentTok{\#\textgreater{} 2 346  850 36.34     0}
\CommentTok{\#\textgreater{} 3 346  900 36.35     0}
\CommentTok{\#\textgreater{} 4 346  910 36.42     0}
\CommentTok{\#\textgreater{} 5 346  920 36.55     0}
\end{Highlighting}
\end{Shaded}

It is up to the user to define the appearance of these blocks for LaTeX output.

You may also use: \texttt{.rmdcaution}, \texttt{.rmdimportant}, \texttt{.rmdtip}, or \texttt{.rmdwarning} as the block name.

The R Markdown Cookbook provides more help on how to use custom blocks to design your own callouts: \url{https://bookdown.org/yihui/rmarkdown-cookbook/custom-blocks.html}

\hypertarget{sharing-your-book}{%
\chapter{Sharing your book}\label{sharing-your-book}}

\hypertarget{publishing}{%
\section{Publishing}\label{publishing}}

HTML books can be published online, see: \url{https://bookdown.org/yihui/bookdown/publishing.html}

\hypertarget{pages}{%
\section{404 pages}\label{pages}}

By default, users will be directed to a 404 page if they try to access a webpage that cannot be found. If you'd like to customize your 404 page instead of using the default, you may add either a \texttt{\_404.Rmd} or \texttt{\_404.md} file to your project root and use code and/or Markdown syntax.

\hypertarget{metadata-for-sharing}{%
\section{Metadata for sharing}\label{metadata-for-sharing}}

Bookdown HTML books will provide HTML metadata for social sharing on platforms like Twitter, Facebook, and LinkedIn, using information you provide in the \texttt{index.Rmd} YAML. To setup, set the \texttt{url} for your book and the path to your \texttt{cover-image} file. Your book's \texttt{title} and \texttt{description} are also used.

This \texttt{bs4\_book} provides enhanced metadata for social sharing, so that each chapter shared will have a unique description, auto-generated based on the content.

Specify your book's source repository on GitHub as the \texttt{repo} in the \texttt{\_output.yml} file, which allows users to view each chapter's source file or suggest an edit. Read more about the features of this output format here:

\url{https://pkgs.rstudio.com/bookdown/reference/bs4_book.html}

Or use:

\begin{Shaded}
\begin{Highlighting}[]
\NormalTok{?bookdown}\SpecialCharTok{::}\NormalTok{bs4\_book}
\end{Highlighting}
\end{Shaded}


  \bibliography{book.bib,packages.bib}

\end{document}
